%%
%% INSTRUCTIONS FOR COMPILING THIS DOCUMENT
%%
%% Choose one of the two ``documentclass'' declarations below.
%% Use the ``ps'' one for printing; and the ``pdf'' one for an
%% online slideshow PDF format.
%%
%% You will need a tarball of the ``prosper'' software untarred
%% into a directory called ``prosper'' in this directory. You
%% can download prosper from prosper.sourceforge.net.
%%
%%
\documentclass[ps,azure]{prosper}
%%\documentclass[pdf,slideColor,azure]{prosper}

\hypersetup{pdfpagemode=FullScreen}

\title{The Next Big ThiNG}
\subtitle{A Reflective, Transactional, Concurrent, Functional, Object-Oriented, Distributed Programming Language}
\author{Tony Garnock-Jones (tonyg@kcbbs.gen.nz)\\
Michael Bridgen (mikeb@squaremobius.net)}

\date{30 November 2004}

\begin{document}
\maketitle

\begin{slide}{Introduction}
  This talk will cover
  \begin{itemize}
  \item An introduction to process languages
  \item Applications
  \end{itemize}
\end{slide}

\part{Introduction to Process Languages}

\overlays{5}{
\begin{slide}{What is a process language?}
  A {\bf process language} describes a group of simultaneously
  executing programs and how they communicate.

  \begin{itemstep}
  \item Parallel -- not sequential
  \item Allows solid reasoning about concurrency
  \item Strong mathematical foundations can allow you to {\bf prove}
    that your program behaves in certain ways
  \item This is something traditional languages are poor at -- using
    threads is a nightmare; no math in sight!
  \item Not a panacea -- deadlock still possible, etc.
  \end{itemstep}
\end{slide}
}

\overlays{4}{
\begin{slide}{Why use a process language?}
  Process languages are a natural fit for event-based systems:
  \begin{itemstep}
  \item server-side web applications
  \item telecommunications
  \item operating system kernels
  \item GUIs
  \end{itemstep}
\end{slide}
}

\overlays{5}{
\begin{slide}{Why use a process language?}
  They work well in more general settings, too:
  \begin{itemstep}
  \item coordination of general distributed computing
    \begin{itemize}
    \item SOAP, XML-RPC
    \item CORBA
    \end{itemize}
  \item general modeling of processes
  \item coordination of general business processes (BizTalk)
  \end{itemstep}
\end{slide}
}

\begin{slide}{The $\pi$-calculus}
  There are many kinds of process language. I am most familiar with
  the $\pi$-calculus, so that will be the foundation for this talk.

  \begin{itemize}
  \item {\bf Milner, Parrow and Walker} developed the $\pi$-calculus,
    working from the foundation of {\it CCS}
  \item See the papers ``A Calculus of Mobile Processes'', parts I and
    II (available online)
  \item Also interesting: ``Functions as Processes'' by Milner
  \end{itemize}
\end{slide}

\overlays{7}{
\begin{slide}{The $\pi$-calculus}
  \begin{itemstep}
  \item Same level as the $\lambda$-calculus
    \begin{itemize}
    \item Both Turing-equivalent
    \item $\lambda$ datatype: the {\it function}
    \item $\pi$ datatype: the {\it name} (or {\it port})
    \end{itemize}
  \item Where $\lambda$ {\it applies} functions (to functions), $\pi$
    {\it sends} names (down names).
  \item Neither language needs other types to be complete
  \item However: practical implementations include strings, ints,
    floats, structures, lists, vectors etc.
  \end{itemstep}
\end{slide}
}

\overlays{6}{
\begin{slide}{$\lambda$-calculus combinators}
  The $\lambda$-calculus has two combinators:
  \begin{itemstep}
  \item abstraction:
    \begin{itemize}
    \item $\lambda x . M$
    \item {\tt (lambda (x) M)}
    \end{itemize}
  \item application (curried):
    \begin{itemize}
    \item $M N$
    \item {\tt (M N)}
    \end{itemize}
  \end{itemstep}
\end{slide}
}

\overlays{6}{
\begin{slide}{$\pi$-calculus combinators}
  The $\pi$-calculus has six combinators:
  \begin{itemstep}
  \item new port: $(\nu p) . P$
  \item read: $p(x) . P$
  \item write: $\bar p\left<x\right> . P$
  \item repeat: $\mathbf{!}P$
  \item parallel: $P~|~Q~|~...$
  \item choice (``sum''): $P~+~Q~+~...$
  \end{itemstep}
\end{slide}
}

\overlays{10}{
\begin{slide}{Hidden Machinery}
  \begin{itemstep}
  \item All those combinators!
  \item $\pi$ {\it seems} more complicated, but...
  \item $\lambda$ is hiding a {\bf lot} of machinery
    \begin{itemize}
    \item CPS conversion
    \item replication of process bodies
    \item order of application -- $\lambda _v$ vs $\lambda _n$
    \end{itemize}
  \item compare $\pi$, where it's all in the open
    \begin{itemize}
    \item no analogue to direct-style $\leftrightarrow$ CPS
    \item replication explicit (with `$\mathbf{!}$')
    \item no direct-style $\rightarrow$ no assumptions
    \end{itemize}
  \end{itemstep}
\end{slide}
}

\overlays{4}{
\begin{slide}{Getting $\lambda$ from $\pi$}
  Sequential languages ($\lambda$) are a {\it restriction} of parallel
  languages ($\pi$) -- easy to get to $\lambda$ from $\pi$!

  \begin{itemstep}
  \item {\it closures} map to {\it names} -- the body of the closure
    is translated into a repeated-read from the name
  \item many slightly different ways of embedding $\lambda$ in $\pi$
    \begin{itemize}
    \item the precise calculus makes a difference: $\lambda _ v$ vs $\lambda _ n$ etc.
    \item specific details have a large effect on efficiency
    \end{itemize}
  \end{itemstep}
\end{slide}
}

\overlays{3}{
\begin{slide}{$\pi$ is fine-grained}
  \begin{itemstep}
  \item $\lambda$ reifies {\it some} aspects of control -- after
    conversion to continuation-passing-style!
  \item $\pi$ reifies {\it all} aspects of control -- no rewriting
    required or possible!
  \item $\pi$ also copes naturally with parallelism -- not a layer on
    top
  \end{itemstep}
\end{slide}
}

\overlays{7}{
\begin{slide}{Getting $\pi$ from $\lambda$}
  \begin{itemstep}
  \item Scheme is essentially $\lambda _v$ made real
  \item Scheme + small framework + macros $\longrightarrow \pi$
  \item Required:
    \begin{itemize}
    \item A port data structure
    \item A scheduler
    \item Macros that expand into calls on the above
    \item Unification (optional, for complex data)
    \end{itemize}
  \end{itemstep}
\end{slide}
}

\part{Applications}

\overlays{5}{
\begin{slide}{Web programming}
  \begin{itemstep}
  \item Concurrency -- each request is a separate thread
  \item Using threads complex; inverted flow-of-control
  \item Two potential solutions:
    \begin{itemize}
    \item Scheme-defined {\tt call/cc} microthreads
    \item Use a $\pi$-calculus-based language!
    \end{itemize}
  \end{itemstep}
\end{slide}
}

\overlays{4}{
\begin{slide}{$\pi$ on the web}
  Using a $\pi$-calculus based language for web programming:

  \begin{itemstep}
  \item Each {\it session} is a process on the server
  \item Each web {\it client} is just another port
  \item Any {\it databases} are just ports
  \item A {\it well-known-port} is used to start sessions
  \end{itemstep}
\end{slide}
}

\overlays{3}{
\begin{slide}{Telco -- Switching}
  Important requirements:
  \begin{itemstep}
  \item Massively parallel -- $\geq 10000$ active calls
  \item Highly available -- ``five-nines''
  \item Realtime
  \end{itemstep}
\end{slide}
}

\overlays{6}{
\begin{slide}{Telco -- Switching}
  $\pi$ addresses these requirements:
  \begin{itemstep}
  \item Massively parallel -- $\pi$ processes lightweight
  \item Highly available -- $\pi$ can be strongly typed
    \begin{itemize}
    \item behavioural types capture a protocol definition
    \item programs can be proven to conform to a protocol (cf. threads)
    \end{itemize}
  \item Realtime -- that depends!
    \begin{itemize}
    \item usually waiting for outside databases anyway...
    \end{itemize}
  \end{itemstep}
\end{slide}
}

\overlays{4}{
\begin{slide}{Telco -- IN}
  \begin{itemstep}
  \item Specific telephony application: ``Intelligent Networking'' (IN)
  \item Advanced network functionality -- voicemail, menus, recharge,
    roaming, call gapping, 3-way calls ...
  \item Many interlocking, massively-parallel pieces
  \item Protocol-driven design (perfect for $\pi$!)
  \end{itemstep}
\end{slide}
}

\overlays{8}{
\begin{slide}{Telco -- IN}
  \begin{itemstep}
  \item Many components in an IN network:
    \begin{itemize}
    \item SSP -- the switch
    \item SCP -- application host
    \item SMP -- management
    \item IP -- peripheral (announcements etc.)
    \item BE -- billing engines
    \end{itemize}
  \item Each one with many open calls
  \item Each one (usually) with a state-machine per call
  \end{itemstep}
\end{slide}
}

\begin{slide}{Example Call}
  Complicated protocols -- I don't remember full detail!
  \begin{itemize}
  \item IDP from SSP $\longrightarrow$ SCP
  \item call gapping may count the IDP
  \item may connect to IP for voice prompts, collect digits
  \item complex applications may be built
  \item may connect to databases, external BEs for credit checks
  \item may either connect, disconnect or reroute the call
  \end{itemize}
\end{slide}

\begin{slide}{Telco -- IN}
  Programming models:
  \begin{itemize}
  \item State machines using C, C++ etc
    \begin{itemize}
    \item traditional
    \item error-prone
    \item unintuitive
    \item not compositional
    \item does not distribute
    \end{itemize}
  \end{itemize}
\end{slide}

\begin{slide}{Telco -- IN}
  Programming models:
  \begin{itemize}
  \item Threading
    \begin{itemize}
    \item doesn't scale
    \item error-prone (preemptive threads)
    \item intuitive (modulo locking)
    \item coarse-grained, heavyweight
    \item does not distribute
    \end{itemize}
  \end{itemize}
\end{slide}

\begin{slide}{Telco -- IN}
  Programming models:
  \begin{itemize}
  \item Scheme {\small (or Python!)} microthreads
    \begin{itemize}
    \item scalable
    \item intuitive
    \item medium-fine-grained, lightweight
    \item slightly awkward -- layer on top
    \item does not distribute trivially
    \end{itemize}
  \end{itemize}
\end{slide}

\begin{slide}{Telco -- IN}
  Programming models:
  \begin{itemize}
  \item $\pi$
    \begin{itemize}
    \item scalable
    \item intuitive
    \item fine-grained, lightweight
    \item naturally concurrent
    \item distributes for free!
    \end{itemize}
  \end{itemize}
\end{slide}

\overlays{5}{
\begin{slide}{$\pi$ as a kernel}
  \begin{itemstep}
  \item Very close to the hardware
  \item Safe -- behavioural types
  \item Massively multithreaded -- SMP
  \item Message passing -- distribution -- plug and play supercomputer!
  \item IRQ handling sketch
  \end{itemstep}
\end{slide}
}

\begin{slide}{Using $\pi$ for GUIs}
  \begin{itemize}
  \item Similar to web programming
  \item A little more fine grained, slightly different focus
  \item M-V-C coordination
  \end{itemize}
\end{slide}

\overlays{2}{
\begin{slide}{Coordination}
  $\pi$ scales from micro- to macro-programming. It has a place in:
  \begin{itemize}
  \item protocols at the IRQ level
  \item protocols at the OS level
  \item protocols at the network level
  \item protocols at the application level
  \end{itemize}

  \FromSlide{2}
  ... so why not protocols between applications, too?
  \begin{itemize}
  \item glue between local apps
  \item glue across the internet
  \end{itemize}
\end{slide}
}

\overlays{4}{
\begin{slide}{Local $\pi$ glue}
  The ultimate scripting language -- sequential when required,
  parallel when you like!

  \begin{itemstep}
  \item {\tt expect}
  \item Local CORBA or COM
  \item Calling out to .NET
  \item Interoperating with local Java
  \end{itemstep}
\end{slide}
}

\overlays{4}{
\begin{slide}{Remote $\pi$ glue}
  Synchronising and scheduling many distributed applications in the
  right order -- ``unified messaging''
  \begin{itemstep}
  \item Encode business logic
  \item Remote CORBA or DCOM
  \item SOAP, XML-RPC
  \item .NET remoting -- similar in ambition?
  \end{itemstep}
\end{slide}
}

\overlays{4}{
\begin{slide}{Modeling using $\pi$}
  The $\pi$-calculus naturally covers an impressive subset of the jobs
  UML is used for:

  \begin{itemstep}
  \item It's fine-grained $\therefore$ everything is explicit
  \item Normal code sequential -- highlights control flow
  \item You can {\it reproject} code to highlight dataflow
  \item Message sequence diagrams can be derived automatically
  \end{itemstep}
\end{slide}
}

\overlays{3}{
\begin{slide}{B2B}
  Extend the scope of the model to include ``real-world'' processes
  \begin{itemstep}
  \item mail-order
  \item project management processes
  \item paper-based records
  \end{itemstep}
\end{slide}
}

\overlays{3}{
\begin{slide}{B2B}
  B2B systems based on, or related to process-language ideas:
  \begin{itemstep}
  \item BizTalk (Microsoft) -- XML-based B2B glue
  \item WSDL (W3C, IBM, Microsoft) -- another XML-based B2B glue
  \item others? {\small (not my field...)}
  \end{itemstep}
\end{slide}
}

\part{The End}

%%%%%%%%%%%%%%%%%%%%%%%%%%%%%%%%%%%%%%%%%%%%%%%%%%%%%%%%%%%%%%%%%%%%%%%%%%%

\part{Backup}

\begin{slide}{Getting $\lambda$ from $\pi$}
  One way of translating abstraction:

 \[
 \left[\lambda x.M \right]_k
 \longrightarrow
 (\nu f)~\mathbf{!}f(k',x) . \left( \left[M\right]_{k'} | \bar k \left<f\right> \right)
 \]

 Translating application is different for $\lambda _v$ vs $\lambda _n$.

 {\small (This is just to convey the flavour an embedding has -- this
   particular definition may not be 100\% correct in all situations!)}
\end{slide}

%%%%%%%%%%%%%%%%%%%%%%%%%%%%%%%%%%%%%%%%%%%%%%%%%%%%%%%%%%%%%%%%%%%%%%%%%%%

%% \overlays{1}{
%% \begin{slide}{}

%%   \begin{itemstep}
%%   \item foo
%%   \end{itemstep}
%% \end{slide}
%% }

\end{document}
